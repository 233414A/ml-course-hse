\documentclass[12pt,fleqn]{article}
%\usepackage{./vkCourseML}
\usepackage{../lecture-notes/vkCourseML}
\usepackage{lipsum}
\usepackage{indentfirst}
\usepackage{graphicx}
\graphicspath{{./figures/}}
\title{Машинное обучение, ФКН ВШЭ\\Семинар №9}
\author{}
\date{8 ноября 2019 г.}

\begin{document}
\maketitle


\section{Градиентный бустинг}

Сегодня мы обсудим один из способов построения композиций алгоритмов - градиентный бустинг. Будем искать алгоритм, оптимизирующий некоторую дифференцируемую функцию потерь $L(y, z)$, в виде взвешенной суммы базовых алгоритмов:
\[
    a_N(x)
    =
    \sum_{n = 0}^{N}
        \gamma_n b_n(x).
\]

Идея бустинга заключается в последовательном построении алгоритмов, каждый из которых учитывает ошибки построенной до сих пор композиции:
\[
    \sum_{i = 1}^{\ell}
        L(y_i, a_{N - 1}(x_i) + \gamma_N b_N(x_i))
    \to
    \min_{b_N, \gamma_N}.
\]

После выбора каких-нибудь простых $\gamma_0$ и $b_0(x)$ (например, для задачи регрессии можно положить $\gamma_0 = 1$ и $b_0(x) = \frac 1\ell \sum_{i=1}^l y_i$) все последующие базовые алгоритмы стараются приблизить антиградиент суммы функций потерь, взятый в точках $z = a_{N - 1}(x_i)$:
\[
    s_i
    =
    -
    \left.
    \frac{\partial L}{\partial z}
    \right|_{z = a_{N - 1}(x_i)}.
\]

При этом приближается антиградиент с точки зрения квадратичной функции потерь:
\[
    b_N(x)
    =
    \argmin_{b \in \AA}
        \sum_{i = 1}^{\ell}
            \left(
                b(x_i) - s_i
            \right)^2.
\]

Подбор коэффициентов производится по аналогии с методом наискорейшего спуска:
\[
    \gamma_N = \argmin_{\gamma \in \mathbb{R}}\sum_{i=1}^{\ell} L(y_i, a_{N-1}(x_i) + 
        \gamma b_N (x_i)).
\]

Мы уже знаем, что для квадратичной функции потерь $L(y, z) = 
(y - z)^2$ сдвиги $s_i^*$ выражаются как $s_i^* = y_i - a_N(x_i)$. Посмотрим, 
чему равны сдвиги для других функций потерь.

\begin{vkProblem}
Найдите сдвиги для функции потерь $L(y, z) = |y - z|$.
\end{vkProblem}
\begin{esSolution}

\[
    s_i = - \left. \frac{\partial |y_i - z|}{\partial 
        z} \right|_{z=a_{N-1}(x_i)} = 
    \Bigl. \sign(y_i - z) \Bigr|_{z=a_{N-1}(x_i)} =
    \sign(y_i - a_{N-1}(x_i))
\]

\end{esSolution}

\begin{vkProblem}
Найдите сдвиги для логистической функции потерь \\
$L(y, z) = \log (1 + \exp(-yz))$.
\end{vkProblem}
\begin{esSolution}
Вспомним, что логистическая функция потерь выражается через сигмоиду $\sigma(x) 
= \frac{1}{1 + \exp(-x)}$ следующим образом:

\[
    L(y, z) = \log \left( \frac{1}{\sigma(yz)} \right) = -\log \sigma(yz)
\]

Далее, пользуясь формулой для производной сигмоиды $\sigma'(x) = \sigma(x)(1 - 
\sigma(x))$, получаем

\[
    s_i = \left. \frac{\partial \log \sigma(y_i z)}{\partial z} 
        \right|_{z=a_{N-1}(x_i)} =
    \left. \frac{1}{\sigma(y_i z)} \sigma(y_i z) (1 - \sigma(y_i z)) y_i
        \right|_{z=a_{N-1}(x_i)} =
\]
\[
    = (1-\sigma(y_i a_{N-1}(x_i))) y_i =
    \frac{y_i}{1 + \exp(y_i a_{N-1}(x_i))}.
\]

\end{esSolution}

Теперь посмотрим, как выглядят оптимальные коэффициенты для квадратичной функции потерь:

\begin{vkProblem}
Предположим, что для квадратичной функции потерь $ L$ на очередном шаге бустинга сдвиги для обучения равны $s_i$, и на этих сдвигах был обучен алгоритм $b_N(x)$. Найдите 
оптимальное значение веса алгоритма $\gamma_N$.
\end{vkProblem}
\begin{esSolution}

\[
    \gamma_N = \argmin_{\gamma \in \mathbb{R}} \sum_{i=1}^{\ell} L(y_i, a_{N-1}(x_i) + 
        \gamma b_N (x_i)) =
    \argmin_{\gamma \in \mathbb{R}}\sum_{i=1}^\ell (y_i - (a_{N-1}(x_i) + \gamma b_N (x_i)))^2 =
\]

\[
    = \argmin_{\gamma \in \mathbb{R}}\sum_{i=1}^\ell (s_i - \gamma b_N (x_i)))^2.
\]

Заметим, что функционал является выпуклой функцией относительно $\gamma$, поэтому для нахождения минимума продифференцируем функцию по $\gamma$:

\[
    \frac{\partial \left(\sum_{i=1}^\ell (s_i - \gamma b_N (x_i)))^2 \right) }
        {\partial \gamma} =
    \sum_{i=1}^\ell \frac{\partial (s_i - \gamma b_N (x_i)))^2}{\partial 
    \gamma} =
    \sum_{i=1}^\ell (s_i - \gamma b_N(x_i)) (- b_N(x_i)) =
\]

\[
    = \gamma \sum_{i=1}^\ell b_N(x_i)^2 - \sum_{i=1}^\ell s_i b_N(x_i).
\]

Приравнивая результат к нулю, получаем

\[
    \gamma = \frac{\sum_{i=1}^\ell s_i b_N(x_i)}{\sum_{i=1}^\ell b_N(x_i)^2}.
\]

\end{esSolution}


\section{Градиентный бустинг над деревьями}

С точки зрения разложения на смещение и разброс в ходе обучения каждого нового алгоритма уменьшается смещение композиции, а разброс же остаётся неизменным или увеличивается. Поэтому на практике чаще всего в качестве базовых алгоритмов используются деревья небольшой грубины, обладающие большим смещением, но не склонные к переобучению. Мы знаем, что решающее дерево разбивает все пространство на непересекающиеся области, в каждой из которых его ответ равен константе:
\[
    b_n(x)
    =
    \sum_{j = 1}^{J_n}
        b_{nj}
        [x \in R_j],
\]
где~$j = 1, \dots, J_n$~--- индексы листьев,
$R_j$~--- соответствующие области разбиения,
$b_{nj}$~--- значения в листьях.

Значит, на~$N$-й итерации бустинга композиция обновляется как
\[
    a_N(x)
    =
    a_{N - 1}(x)
    +
    \gamma_N
    \sum_{j = 1}^{J_N}
        b_{Nj}
        [x \in R_j].
\]
Видно, что добавление в композицию одного дерева с~$J_N$ листьями равносильно
добавлению~$J_N$ базовых алгоритмов, представляющих собой предикаты.
Мы можем улучшить качество композиции, подобрав свой коэффициент
при каждом из предикатов:
\[
    \sum_{i = 1}^{\ell}
        L\left(
            y_i,
            a_{N - 1}(x_i)
            +
            \sum_{j = 1}^{J_N}
                \gamma_{Nj}
                [x \in R_j]
        \right)
    \to
    \min_{\{\gamma_{Nj}\}_{j = 1}^{J_N}}.
\]
Поскольку области разбиения~$R_j$ не пересекаются,
данная задача распадается на~$J_N$ независимых подзадач:
\[
    \gamma_{Nj}
    =
    \argmin_\gamma
        \sum_{x_i \in R_j}
            L(y_i, a_{N - 1}(x_i) + \gamma),
    \qquad
    j = 1, \dots, J_N.
\]

Далее на семинаре мы будем сравнивать градиентный бустинг и случайные леса. Для того, чтобы сравнивать сложность построения композиций, получаемых этими методами, нам потребуется решить следующую задачу:

\begin{vkProblem}
Дана выборка из $\ell$ объектов, описываемых $d$ признаками. Приведите временную асимптотику обучения и построения прогнозов для композиции вида $a_N(x) = \sum_{n=0}^N b_n(x)$ над решающими деревьями $b_n$ глубины не более, чем $D$.
\end{vkProblem}
\begin{esSolution} 
На стадии обучения при построении очередного решающего дерева глубины $D$ нам необходимо выбрать предикаты в не более чем $2^D - 1$ внутренних вершинах этого дерева. Каждый порог выбирается путем перебора $\le \ell - 1$ значений  для каждого из $d$ признаков. Для расчета прогнозов нужно распределить объекты обучающей выборки по листьям за линейное от количества объектов и предикатов время. После выбора предиктов необходимо вычислить прогноз дерева в каждом из листов за линейное время от количества объектов обучающей выборки, попавших в лист - для вычисления прогнозов во всех листах одновременно нам достаточно один раз пройтись по всем объектам. Отсюда асимптотика построения одного решающего дерева — $O(2^D \ell d + \ell D + \ell) = O(2^Dd \ell ).$

На стадии построения прогноза для объекта $x$ он «пропускается» через дерево от корня к листьям, тем самым проходя путь из не более чем $D$ внутренних вершин, в каждой из которых происходит проверка предиката за константное время. Отсюда имеем асимптотику для построения прогноза композиции — $O(ND)$.

\end{esSolution}

Таким образом, вычислительно менее сложным оказывается процесс обучения большого количества неглубоких деревьев, а потому градиентный бустинг является более выгодным с точки зрения времени обучения алгоритмом по сравнению со случайным лесом.

Заметим также, что, поскольку обучение градиентного бустинга является «направленным», то ему требуется меньшее по сравнению со случайным лесом количество базовых алгоритмов для достижения того же качества композиции.

\end{document}
