\documentclass[12pt,fleqn]{article}
\usepackage{vkCourseML}
\hypersetup{unicode=true}
%\usepackage[a4paper]{geometry}
\usepackage[hyphenbreaks]{breakurl}

\interfootnotelinepenalty=10000

\begin{document}
\title{Лекция 2\\Линейная регрессия}
\author{Е.\,А.\,Соколов\\ФКН ВШЭ}
\maketitle

\section{Линейные модели}

На предыдущей лекции мы уже упоминали линейные регрессионные модели.
Такие модели сводятся к суммированию значений признаков с некоторыми весами:

\begin{equation}
\label{eq:linear}
    a(x)
    =
    w_0
    +
    \sum_{j = 1}^{d}
        w_j x_j.
\end{equation}
Параметрами модели являются~\emph{веса} или~\emph{коэффициенты}~$w_j$.
Вес~$w_0$ также называется свободным коэффициентом или \emph{сдвигом}~(bias).
Заметим, что сумма в формуле~\eqref{eq:linear} является скалярным произведением
вектора признаков на вектор весов.
Воспользуемся этим и запишем линейную модель в более компактном виде:
\begin{equation}
\label{eq:linearCompact}
    a(x)
    =
    w_0
    +
    \langle w, x \rangle,
\end{equation}
где~$w = (w_1, \dots, w_d)$~--- вектор весов.

Достаточно часто используется следующий приём, позволяющий упростить запись ещё сильнее.
Добавим к признаковому описанию каждого объекта $(d + 1)$-й признак, равный единице.
Вес при этом признаке как раз будет иметь смысл свободного коэффициента,
и необходимость в слагаемом~$w_0$ отпадёт:
\[
    a(x)
    =
    \langle w, x \rangle.
\]
Тем не менее, при такой форме следует соблюдать осторожность и помнить
о наличии в выборке специального признака.
Например, мы столкнёмся со сложностями, связанными с этим, когда будем говорить о регуляризации.

За счёт простой формы линейные модели достаточно быстро и легко обучаются,
и поэтому популярны при работе с большими объёмами данных.
Также у них мало параметров, благодаря чему удаётся контролировать риск переобучения
и использовать их для работы с зашумлёнными данными и с небольшими выборками.

\section{Области применимости линейных моделей}

Сложно представить себе ситуацию, в которой мы берём данные, обучаем линейную модель и получаем хорошее качество работы.
В линейной модели предполагается конкретный вид зависимости~--- а именно, что каждый признак линейно влияет на целевую переменную,
и что целевая переменная не зависит от каких-либо комбинаций признаков.
Вряд ли это будет выполнено по умолчанию, поэтому обычно данные требуют специальной подготовки, чтобы линейные модели
оказались адекватными задаче.
Приведём несколько примеров.

\paragraph{Категориальные признаки.}
Представим себе задачу определения стоимости квартиры по её характеристикам.
Одним из важных признаков является район, в котором находится квартира.
Этот признак является категориальным~--- его значения нельзя сравнивать между собой на больше/меньше,
их нельзя складывать или вычитать.
Непосредственно такие признаки нельзя использовать в линейных моделях, но есть
достаточно распространённый способ их преобразования.

Допустим, категориальный признак~$f_j(x)$ принимает значения из множества~$C = \{c_1, \dots, c_m\}$.
Заменим его на~$m$ бинарных признаков~$b_1(x), \dots, b_m(x)$, каждый из которых
является индикатором одного из возможных категориальных значений:
\[
    b_i(x) = [f_j(x) = c_i].
\]
Такой подход называется one-hot кодированием.

Отметим, что признаки~$b_1(x), \dots, b_m(x)$ являются линейно зависимыми:
для любого объекта выполнено
\[
    b_1(x) + \dots + b_m(x) = 1.
\]
Чтобы избежать этого, можно выбрасывать один из бинарных признаков.
Впрочем, такое решение имеет и недостатки~--- например, если на тестовой выборке
появится новая категория, то её как раз можно закодировать с помощью нулевых бинарных признаков;
при удалении одного из них это потеряет смысл.

Вернёмся к задаче про стоимость квартиры.
Если мы применим линейную модель к данным после one-hot кодирования признака о районе~(допустим, это $f(x)$), то получится
такая формула:
\[
    a(x)
    =
    w_1 [f(x) = c_1]
    +
    \dots
    +
    w_m [f(x) = c_m]
    +
    \{
        \text{взаимодействие с другими признаками}
    \}.
\]
Такая зависимость кажется логичной~--- каждый район задаёт некоторый базовый уровень
стоимости~(например, для района~$c_1$ имеем базовую цену~$w_1$),
а остальные факторы корректируют его.

\paragraph{Работа с текстами.}

Перейдём к предсказанию стоимости квартиры по её текстовому описанию.
Есть простой способ кодирования, который называется~\emph{мешок слов~(bag of words)}.

Найдём все слова, которые есть в нашей выборке текстов, и пронумеруем их:~$\{c_1, \dots, c_m\}$.
Будем кодировать текст~$m$ признаками~$b_1(x), \dots, b_m(x)$, где~$b_j(x)$ равен
количеству вхождений слова~$c_j$ в текст.
Линейная модель над такими признаками будет иметь вид
\[
    a(x)
    =
    w_1 b_1(x)
    +
    \dots
    +
    w_m b_m(x)
    +
    \dots,
\]
и такой вид тоже кажется разумным.
Каждое вхождение слова~$c_j$ меняет прогноз стоимости на~$w_j$.
В самом деле, можно ожидать, что слово~<<престижный>> скорее говорит о том,
что квартира дорогая, а слово~<<плохой>> вряд ли будут использовать при описании
приличной квартиры.

\paragraph{Бинаризация числовых признаков.}

Наконец, подумаем о предсказании стоимости квартиры по расстоянию до ближайшей станции метро~$x_j$.
Может оказаться, что самые дорогие квартиры расположены где-то в 5-10 минутах ходьбы от метро,
а те, что ближе или дальше, стоят не так дорого.
В этом случае зависимость целевой переменной от признака не будет линейной.
Чтобы сделать линейную модель подходящей, мы можем бинаризовать признак.
Для этого выберем некоторую сетку точек~$\{t_1, \dots, t_m\}$.
Это может быть равномерная сетка между минимальным и максимальным значением признака или,
например, сетка из эмпирических квантилей.
Добавим сюда точки~$t_0 = -\infty$ и~$t_{m+1} = +\infty$.
Новые признаки зададим как
\[
    b_i(x)
    =
    [t_{i - 1} < x_j \leq t_{i}],
    \quad
    i = 1, \dots, m+1.
\]
Линейная модель над этими признаками будет выглядеть как
\[
    a(x)
    =
    w_1 [t_{0} < x_j \leq t_{1}]
    +
    \dots
    +
    w_{m+1} [t_{m} < x_j \leq t_{m+1}]
    +
    \dots,
\]
то есть мы найдём свой прогноз стоимости квартиры для каждого интервала расстояния до метро.
Такой подход позволит учесть нелинейную зависимость между признаком и целевой переменной.

\section{Измерение ошибки в задачах регрессии}

Чтобы обучать регрессионные модели, нужно определиться, как именно измеряется качество предсказаний.
Будем обозначать через~$y$ значение целевой переменной, через~$a$~--- прогноз модели.
Рассмотрим несколько способов оценить отклонение~$L(y, a)$ прогноза от истинного ответа.

\paragraph{MSE и $R^2$.}

Основной способ измерить отклонение~--- посчитать квадрат разности:
\[
    L(y, a) = (a - y)^2
\]
Благодаря своей дифференцируемости эта функция наиболее часто используется в задачах регрессии.
Основанный на ней функционал называется среднеквадратичным отклонением~(mean squared error, MSE):
\[
    \text{MSE}(a, X)
    =
    \frac{1}{\ell}
    \sum_{i = 1}^{\ell} \left(
        a(x_i) - y_i
    \right)^2.
\]
Отметим, что величина среднеквадратичного отклонения плохо интерпретируется,
поскольку не сохраняет единицы измерения~--- так, если мы предсказываем цену
в рублях, то MSE будет измеряться в квадратах рублей.
Чтобы избежать этого, используют корень из среднеквадратичной ошибки~(root mean squared error, RMSE):
\[
    \text{RMSE}(a, X)
    =
    \sqrt{
        \frac{1}{\ell}
        \sum_{i = 1}^{\ell} \left(
            a(x_i) - y_i
        \right)^2
    }.
\]

Среднеквадратичная ошибка подходит для сравнения двух моделей
или для контроля качества во время обучения,
но не позволяет сделать выводы о том, насколько хорошо данная модель
решает задачу.
Например, $\text{MSE}=10$ является очень плохим показателем,
если целевая переменная принимает значения от 0 до 1,
и очень хорошим, если целевая переменная лежит в интервале~$(10000, 100000)$.
В таких ситуациях вместо среднеквадратичной ошибки полезно использовать~\emph{коэффициент детерминации}
(или коэффициент~$R^2$):
\[
    R^2(a, X)
    =
    1
    -
    \frac{
        \sum_{i = 1}^{\ell} (a(x_i) - y_i)^2
    }{
        \sum_{i = 1}^{\ell} (y_i - \bar y)^2
    },
\]
где~$\bar y = \frac{1}{\ell} \sum_{i = 1}^{\ell} y_i$~--- среднее значение целевой переменной.
Коэффициент детерминации измеряет долю дисперсии, объяснённую моделью, в общей дисперсии
целевой переменной.
Фактически, данная мера качества~--- это нормированная среднеквадратичная ошибка.
Если она близка к единице, то модель хорошо объясняет данные,
если же она близка к нулю, то прогнозы сопоставимы по качеству с константным предсказанием.

\paragraph{MAE.}

Заменим квадрат отклонения на модуль:
\[
    L(y, a) = |a - y|
\]
Соответствующий функционал называется средним абсолютным отклонением~(mean absolute error, MAE):
\[
    \text{MAE}(a, X)
    =
    \frac{1}{\ell}
    \sum_{i = 1}^{\ell} \left|
        a(x_i) - y_i
    \right|.
\]

Модуль отклонения не является дифференцируемым, но при этом менее чувствителен к выбросам.
Квадрат отклонения, по сути, делает особый акцент на объектах с сильной ошибкой,
и метод обучения будет в первую очередь стараться уменьшить отклонения на таких объектах.
Если же эти объекты являются выбросами~(то есть значение целевой переменной на них либо ошибочно,
либо относится к другому распределению и должно быть проигнорировано),
то такая расстановка акцентов приведёт к плохому качеству модели.
Модуль отклонения в этом смысле гораздо более терпим к сильным ошибкам.

Приведём ещё одно объяснение того, почему модуль отклонения устойчив к выбросам,
на простом примере.
Допустим, все~$\ell$ объектов выборки имеют одинаковые признаковые описания, но разные
значения целевой переменной~$y_1, \dots, y_\ell$.
В этом случае модель должна на всех этих объектах выдать один и тот же ответ.
Если мы выбрали MSE в качестве функционала ошибки, то получаем следующую задачу:
\[
    \frac{1}{\ell}
    \sum_{i = 1}^{\ell} \left(
        a - y_i
    \right)^2
    \to
    \min_a
\]
Легко показать, что минимум достигается на среднем значении всех ответов:
\[
    a_{\text{MSE}}^*
    =
    \frac{1}{\ell}
    \sum_{i = 1}^{\ell}
        y_i.
\]
Если один из ответов на порядки отличается от всех остальных~(то есть является выбросом),
то среднее будет существенно отклоняться в его сторону.

Рассмотрим теперь ту же ситуацию, но с функционалом MAE:
\[
    \frac{1}{\ell}
    \sum_{i = 1}^{\ell} \left|
        a - y_i
    \right|
    \to
    \min_a
\]
Теперь решением будет медиана ответов:
\[
    a_{\text{MAE}}^*
    =
    \text{median}
        \{y_i\}_{i = 1}^{\ell}.
\]
Небольшое количество выбросов никак не повлияет на медиану~--- она существенно
более устойчива к величинам, выбивающимся из общего распределения.

\paragraph{MSLE.}

Перейдём теперь к логарифмам ответов и прогнозов:
\[
    L(y, a) = (\log(a + 1) - \log(y + 1))^2
\]
Соответствующий функционал называется среднеквадратичной логарифмической ошибкой~(mean
squared logarithmic error, MSLE).
Данная метрика подходит для задач с неотрицательной целевой переменной.
За счёт логарифмирования ответов и прогнозов мы скорее штрафуем за отклонения
в порядке величин, чем за отклонения в их значениях.
Также следует помнить, что логарифм не является симметричной функцией,
и поэтому данная функция потерь штрафует заниженные прогнозы сильнее,
чем завышенные.

\paragraph{MAPE и SMAPE.}

В задачах прогнозирования обычно измеряется относительная ошибка:
\[
    L(y, a) = \left| \frac{y - a}{y} \right|
\]
Соответствующий функционал называется средней абсолютной процентной ошибкой~(mean
absolute percentage error, MAPE).
Данный функционал часто используется в задачах прогнозирования.
Также используется его симметричная модификация~(symmetric mean absolute percentage error, SMAPE):
\[
    L(y, a) = \frac{|y - a|}{(|y| + |a|) / 2}
\]

\section{Обучение линейной регрессии}

Чаще всего линейная регрессия обучается с использованием среднеквадратичной ошибки.
В этом случае получаем задачу оптимизации~(считаем, что среди признаков есть константный, и поэтому
свободный коэффициент не нужен):
\[
    \frac{1}{\ell}
    \sum_{i = 1}^{\ell} \left(
        \langle w, x_i \rangle - y_i
    \right)^2
    \to
    \min_{w}
\]

Эту задачу можно переписать в матричном виде.
Если~$X$~--- матрица <<объекты-признаки>>, $y$~--- вектор ответов, $w$~--- вектор параметров,
то приходим к виду
\begin{equation}
\label{eq:lsq}
    \frac{1}{\ell}
    \left\|
        Xw - y
    \right\|^2
    \to
    \min_{w},
\end{equation}
где используется обычная~$L_2$-норма.
Если продифференцировать данный функционал по вектору~$w$, приравнять к нулю
и решить уравнение, то получим явную формулу для решения~(подробный вывод формулы можно
найти в материалах семинаров):
\[
    w
    =
    (X^T X)^{-1} X^T y.
\]

Безусловно, наличие явной формулы для оптимального вектора весов~--- это большое
преимущество линейной регрессии с квадратичным функционалом.
Но данная формула не всегда применима по ряду причин:
\begin{itemize}
    \item Обращение матрицы~--- сложная операция с кубической сложностью от количества признаков.
        Если в выборке тысячи признаков, то вычисления могут стать слишком трудоёмкими.
        Решить эту проблему можно путём использования численных методов оптимизации.
    \item Матрица~$X^T X$ может быть вырожденной или плохо обусловленной.
        В этом случае обращение либо невозможно, либо может привести к неустойчивым результатам.
        Проблема решается с помощью регуляризации, речь о которой пойдёт ниже.
\end{itemize}

Следует понимать, что аналитические формулы для решения довольно редки в машинном обучении.
Если мы заменим MSE на другой функционал, то найти такую формулу, скорее всего, не получится.
Желательно разработать общий подход, в рамках которого можно обучать модель для широкого
класса функционалов.
Такой подход действительно есть для дифференцируемых функций~--- обсудим его подробнее.

\section{Градиентный спуск и оценивание градиента}

Оптимизационные задачи вроде~\eqref{eq:lsq} можно решать итерационно
с помощью градиентных методов~(или же методов, использующих
как градиент, так и информацию о производных более высокого порядка).

%\subsection{Градиент и его свойства}
\emph{Градиентом} функции~$f: \RR^d \to \RR$ называется вектор его частных производных:
\[
    \nabla f(x_1, \dots, x_d) = \left( \frac{\partial f}{\partial x_j} \right)_{j = 1}^{d}.
\]

Известно, что градиент является направлением наискорейшего роста функции,
а антиградиент~(т.е. $-\nabla f$)~--- направлением наискорейшего убывания.
Это ключевое свойство градиента, обосновывающее его использование в методах оптимизации.

Докажем данное утверждение.
Пусть~$v \in \RR^d$~--- произвольный вектор, лежащий на единичной сфере: $\|v\| = 1$.
Пусть $x_0 \in \RR^d$~--- фиксированная точка пространства.
Скорость роста функции в точке~$x_0$ вдоль вектора~$v$ характеризуется
производной по направлению~$\frac{\partial f}{\partial v}$:
\[
    \frac{\partial f}{\partial v}
    =
    \frac{d}{dt} f(x_{0,1} + t v_1, \dots, x_{0,d} + t v_d) |_{t = 0}.
\]
Из курса математического анализа известно, что данную производную сложной
функции можно переписать следующим образом:
\[
    \frac{\partial f}{\partial v}
    =
    \sum_{j = 1}^{d}
        \frac{\partial f}{\partial x_j}
        \frac{d}{dt} \left(x_{0,j} + t v_j\right)
    =
    \sum_{j = 1}^{d}
        \frac{\partial f}{\partial x_j}
        v_j
    =
    \langle \nabla f, v \rangle.
\]
Распишем скалярное произведение:
\[
    \langle \nabla f, v \rangle
    =
    \|\nabla f\| \|v\| \cos \phi
    =
    \|\nabla f\| \cos \phi,
\]
где~$\phi$~--- угол между градиентом и вектором~$v$.
Таким образом, производная по направлению будет
максимальной, если угол между градиентом и направлением равен нулю,
и минимальной, если угол равен~$180$ градусам.
Иными словами, производная по направлению максимальна
вдоль градиента и минимальна вдоль антиградиента.

У градиента есть ещё одно свойство, которое пригодится нам при попытках
визуализировать процесс оптимизации,~--- он ортогонален линиям уровня.
Докажем это.
Пусть~$x_0$~--- некоторая точка,
$S(x_0) = \{x \in \RR^d \cond f(x) = f(x_0)\}$~--- соответствующая линия уровня.
Разложим функцию в ряд Тейлора на этой линии в окрестности~$x_0$:
\[
    f(x_0 + \eps) = f(x_0) + \langle \nabla f, \eps \rangle + o(\|\eps\|),
\]
где~$x_0 + \eps \in S(x_0)$.
Поскольку~$f(x_0 + \eps) = f(x_0)$~(как-никак, это линия уровня), получим
\[
    \langle \nabla f, \eps \rangle = o(\|\eps\|).
\]
Поделим обе части на~$\|\eps\|$:
\[
    \left\langle \nabla f, \frac{\eps}{\|\eps\|} \right\rangle = o(1).
\]
Устремим~$\|\eps\|$ к нулю.
При этом вектор~$\frac{\eps}{\|\eps\|}$ будет стремится к касательной к линии уровня в точке~$x_0$.
В пределе получим, что градиент ортогонален этой касательной.

\subsection{Градиентный спуск}

Основное свойство антиградиента~--- он указывает в сторону наискорейшего убывания функции в данной точке.
Соответственно, будет логично стартовать из некоторой точки, сдвинуться в сторону антиградиента,
пересчитать антиградиент и снова сдвинуться в его сторону и т.д.
Запишем это более формально.
Пусть~$w^{(0)}$~--- начальный набор параметров~(например, нулевой или сгенерированный из некоторого
случайного распределения).
Тогда градиентный спуск состоит в повторении следующих шагов до сходимости:
\begin{equation}
\label{eq:fullgrad}
    w^{(k)}
    =
    w^{(k - 1)}
    -
    \eta_k
    \nabla Q(w^{(k - 1)}).
\end{equation}
Здесь под~$Q(w)$ понимается значение функционала ошибки для набора параметров~$w$.

Через~$\eta_k$ обозначается длина шага, которая нужна для контроля скорости движения.
Можно делать её константной: $\eta_k = c$.
При этом если длина шага слишком большая, то есть риск постоянно <<перепрыгивать>> через точку минимума,
а если шаг слишком маленький, то движение к минимуму может занять слишком много итераций.
Иногда длину шага монотонно уменьшают по мере движения~--- например, по простой формуле
\[
    \eta_k
    =
    \frac{1}{k}.
\]
В пакете \texttt{vowpal wabbit}, реализующем настройку и применение линейных моделей,
используется более сложная формула для шага в градиентном спуске:
\[
    \eta_k
    =
    \lambda
    \left(
        \frac{s_0}{s_0 + k}
    \right)^p,
\]
где $\lambda$, $s_0$ и~$p$~--- параметры~(мы опустили в формуле множитель, зависящий от номера прохода по выборке).
На практике достаточно настроить параметр~$\lambda$, а остальным
присвоить разумные значения по умолчанию: $s_0 = 1$, $p = 0.5$, $d = 1$.

Останавливать итерационный процесс можно, например, при близости градиента к нулю
или при слишком малом изменении вектора весов на последней итерации.

Если функционал~$Q(w)$ выпуклый, гладкий и имеет минимум~$w^*$,
то имеет место следующая оценка сходимости:
\[
    Q(w^{(k)}) - Q(w^*)
    =
    O(1 / k).
\]

\subsection{Оценивание градиента}

Как правило, в задачах машинного обучения функционал~$Q(w)$ представим в виде суммы~$\ell$ функций:
\[
    Q(w)
    =
    \frac{1}{\ell}
    \sum_{i = 1}^{\ell}
        q_i(w).
\]
В таком виде, например, записан функционал в задаче~\eqref{eq:lsq},
где отдельные функции~$q_i(w)$ соответствуют ошибкам на отдельных объектах.

Проблема метода градиентного спуска~\eqref{eq:fullgrad} состоит в том,
что на каждом шаге необходимо вычислять градиент всей суммы~(будем его называть полным градиентом):
\[
    \nabla_w Q(w)
    =
    \frac{1}{\ell}
    \sum_{i = 1}^{\ell}
        \nabla_w q_i(w).
\]
Это может быть очень трудоёмко при больших размерах выборки.
В то же время точное вычисление градиента может быть не так уж необходимо~---
как правило, мы делаем не очень большие шаги в сторону антиградиента,
и наличие в нём неточностей не должно сильно сказаться на общей траектории.
Опишем несколько способов оценивания полного градиента.

Оценить градиент суммы функций можно градиентом одного случайно взятого слагаемого:
\[
    \nabla_w Q(w)
    \approx
    \nabla_w q_{i_k}(w),
\]
где~$i_k$~--- случайно выбранный номер слагаемого из функционала.
В этом случае мы получим метод~\emph{стохастического
градиентного спуска}~(stochastic gradient descent, SGD)~\cite{robbins51stochastic}:
\[
    w^{(k)} = w^{(k - 1)} - \eta_k \nabla q_{i_k}(w^{(k - 1)}).
\]
Для выпуклого и гладкого функционала может быть получена
следующая оценка:
\[
    \EE \left[
        Q(w^{(k)}) - Q(w^*)
    \right]
    =
    O(1 / \sqrt{k}).
\]
Таким образом, метод стохастического градиента имеет менее
трудоемкие итерации по сравнению с полным градиентом,
но и скорость сходимости у него существенно меньше.

Отметим одно важное преимущество метода стохастического градиентного спуска.
Для выполнения одного шага в данном методе требуется вычислить градиент лишь одного слагаемого~---
а поскольку одно слагаемое соответствует ошибке на одном объекте,
то получается, что на каждом шаге необходимо держать в памяти всего один объект из выборки.
Данное наблюдение позволяет обучать линейные модели на очень больших выборках:
можно считывать объекты с диска по одному, и по каждому делать один шаг метода SGD.

Можно повысить точность оценки градиента, используя несколько слагаемых вместо одного:
\[
    \nabla_w Q(w)
    \approx
    \frac{1}{n}
    \sum_{j = 1}^{n}
    \nabla_w q_{i_{kj}}(w),
\]
где~$q_{i_{kj}}$~--- случайно выбранные номера слагаемых из функционала,
а~$n$~--- параметр метода.
С такой оценкой мы получим метод~mini-batch gradient descent,
который часто используется для обучения дифференцируемых моделей.

В 2013 году был предложен метод~\emph{среднего стохастического градиента}~(stochastic average gradient)~\cite{schmidt13sag},
который в некотором смысле сочетает низкую сложность итераций стохастического градиентного спуска
и высокую скорость сходимости полного градиентного спуска.
В начале работы в нём выбирается первое приближение~$w^0$,
и инициализируются вспомогательные переменные~$z_i^0$,
соответствующие градиентам слагаемых функционала:
\[
    z_i^{(0)}
    =
    \nabla q_i(w^{(0)}),
    \qquad
    i = 1, \dots, \ell.
\]
На~$k$-й итерации выбирается случайное слагаемое~$i_k$ и
обновляются вспомогательные переменные:
\[
    z_i^{(k)}
    =
    \begin{cases}
        \nabla q_i(w^{(k - 1)}),
        \quad
        &\text{если}\ i = i_k;\\
        z_i^{(k - 1)}
        \quad
        &\text{иначе}.
    \end{cases}
\]
Иными словами, пересчитывается один из градиентов слагаемых.
Оценка градиента вычисляется как среднее вспомогательных переменных~---
то есть мы используем все слагаемые, как в полном градиенте,
но при этом почти все слагаемые берутся с предыдущих шагов, а не пересчитываются:
\[
    \nabla_w Q(w)
    \approx
    \frac{1}{\ell}
    \sum_{i = 1}^{\ell}
        z_i^{(k)}.
\]
Наконец, делается градиентный шаг:
\[
    w^{(k)}
    =
    w^{(k - 1)}
    -
    \eta_k
    \frac{1}{\ell}
    \sum_{i = 1}^{\ell}
    z_i^{(k)}.
\]
Данный метод имеет такой же порядок сходимости для выпуклых и гладких функционалов,
как и обычный градиентный спуск:
\[
    \EE \left[
        Q(w^{(k)}) - Q(w^*)
    \right]
    =
    O(1 / k).
\]

Существует множество других способов получения оценки градиента.
Например, это можно делать без вычисления каких-либо градиентов вообще~\cite{flaxman05without}~---
достаточно взять случайный вектор~$u$ на единичной сфере и домножить его
на значение функции в данном направлении:
\[
    \nabla_w Q(w)
    =
    Q(w + \delta u) u.
\]
Можно показать, что данная оценка является несмещённой для сглаженной версии функционала~$Q$.

В задаче оценивания градиента можно зайти ещё дальше.
Если вычислять градиенты~$\nabla_w q_i(w)$ сложно,
то можно~\emph{обучить модель}, которая будет выдавать оценку градиента на основе текущих значений параметров.
Этот подход был предложен для обучения глубинных нейронных сетей~\cite{jaderberg16synthetic}.

\subsection{Модификации градиентного спуска}

С помощью оценок градиента можно уменьшать сложность одного шага градиентного спуска,
но при этом сама идея метода не меняется~--- мы движемся в сторону наискорейшего убывания функционала.
Конечно, такой подход не идеален, и можно по-разному его улучшать, устраняя те или иные его проблемы.
Мы разберём два примера таких модификаций~--- одна будет направлена на борьбу с осцилляциями, а вторая
позволит автоматически подбирать длину шага.

\paragraph{Метод импульса~(momentum).}
Может оказаться, что направление антиградиента сильно меняется от шага к шагу.
Например, если линии уровня функционала сильно вытянуты, то из-за ортогональности градиента линиям уровня
он будет менять направление на почти противоположное на каждом шаге.
Такие осцилляции будут вносить сильный шум в движение, и процесс оптимизации займёт много итераций.
Чтобы избежать этого, можно усреднять векторы антиградиента с нескольких предыдущих шагов~--- в этом
случае шум уменьшится, и такой средний вектор будет указывать в сторону общего направления движения.
Введём для этого вектор инерции:
\begin{align*}
    &h_0 = 0;\\
    &h_k = \alpha h_{k - 1} + \eta_k \nabla_w Q(w^{(k-1)}).
\end{align*}
Здесь~$\alpha$~--- параметр метода, определяющей скорость затухания градиентов с предыдущих шагов.
Разумеется, вместо вектора градиента может быть использована его аппроксимация.
Чтобы сделать шаг градиентного спуска, просто сдвинем предыдущую точку на вектор инерции:
\[
    w^{(k)} = w^{(k-1)} - h_k.
\]

Заметим, что если по какой-то координате градиент постоянно меняет знак, то в результате усреднения
градиентов в векторе инерции эта координата окажется близкой к нулю.
Если же по координате знак градиента всегда одинаковый, то величина соответствующей координаты
в векторе инерции будет большой, и мы будем делать большие шаги в соответствующем направлении.

\paragraph{AdaGrad и RMSprop.}
Градиентный спуск очень чувствителен к выбору длины шага.
Если шаг большой, то есть риск, что мы будем~<<перескакивать>> через точку минимума;
если же шаг маленький, то для нахождения минимума потребуется много итераций.
При этом нет способов заранее определить правильный размер шага~--- к тому же,
схемы с постепенным уменьшением шага по мере итераций могут тоже плохо работать.

В методе AdaGrad предлагается сделать свою длину шага для каждой компоненты вектора параметров.
При этом шаг будет тем меньше, чем более длинные шаги мы делали на предыдущих итерациях:
\begin{align*}
    &G_{kj} = G_{k-1,j} + (\nabla_w Q(w^{(k-1)}))_j^2;\\
    &w_j^{(k)} = w_j^{(k-1)} - \frac{\eta_t}{\sqrt{G_{kj} + \eps}} (\nabla_w Q(w^{(k-1)}))_j.
\end{align*}
Здесь~$\eps$~--- небольшая константа, которая предотвращает деление на ноль.
В данном методе можно зафксировать длину шага~(например,~$\eta_k = 0.01$)
и не подбирать её в процессе обучения.
Отметим, что данный метод подходит для разреженных задач, в которых у каждого объекта большинство признаков равны нулю.
Для признаков, у которых ненулевые значения встречаются редко, будут делаться большие шаги;
если же какой-то признак часто является ненулевым, то шаги по нему будут небольшими.

У метода AdaGrad есть большой недостаток: переменная~$G_{kj}$ монотонно растёт,
из-за чего шаги становятся всё медленнее и могут остановиться ещё до того,
как достигнут минимум функционала.
Проблема решается в методе RMSprop, где используется экспоненциальное затухание градиентов:
\[
    G_{kj} = \alpha G_{k-1,j} + (1 - \alpha) (\nabla_w Q(w^{(k-1)}))_j^2.
\]
В этом случае размер шага по координате зависит в основном от того, насколько
быстро мы двигались по ней на последних итерациях.

\begin{thebibliography}{1}
\bibitem{robbins51stochastic}
    \emph{Robbins, H., Monro S.} (1951).
    A stochastic approximation method.~//
    Annals of Mathematical Statistics,
    22 (3), p. 400-407.
\bibitem{schmidt13sag}
    \emph{Schmidt, M., Le Roux, N., Bach, F. } (2013).
    Minimizing finite sums with the stochastic average gradient.~//
    Arxiv.org.
\bibitem{flaxman05without}
    \emph{Flaxman, Abraham D. and Kalai, Adam Tauman and McMahan, H. Brendan} (2005).
    Online Convex Optimization in the Bandit Setting: Gradient Descent Without a Gradient.~//
    Proceedings of the Sixteenth Annual ACM-SIAM Symposium on Discrete Algorithms.
\bibitem{jaderberg16synthetic}
    \emph{Jaderberg, M. et. al} (2016).
    Decoupled Neural Interfaces using Synthetic Gradients.~//
    Arxiv.org.
\end{thebibliography}

\end{document}
