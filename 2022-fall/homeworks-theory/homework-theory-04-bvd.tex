\documentclass[12pt,fleqn]{article}

\usepackage{vkCourseML}

\usepackage{lipsum}
\usepackage{indentfirst}
\title{Машинное обучение\\ФКН ВШЭ\\Теоретическое домашнее задание №4}
\author{}
\date{}
\theorembodyfont{\rmfamily}
\newtheorem{esProblem}{Задача}

\begin{document}

\maketitle

\begin{esProblem}
    Предположим, что целевая переменная $y$ независима с признаками объекта. Докажите, что в таком случае дисперсия $\text{Var}[y]$ является нижней оценкой квадратичной ошибки любой модели.
\end{esProblem}

\begin{esProblem}
    Допустим, объекты описываются единственным признаком $x \in \mathbb{R}$, имеющим распределение $p$. Рассмотрим некоторую функцию $f(x)$, представимую рядом Тейлора в окрестности нуля. Запишем его: $f(x) = a(x) + \overline{o}(x^k)$, где $a(x)$ - многочлен степени не выше $k$. Пусть целевая переменная определена как $\mathbb{E}[y|x] = f(x)$. Возьмем многочлен $a(x)$ в качестве модели для регрессии $y$. Найдите смещение такой модели для следующих функций и распределений:
    
    \begin{enumerate}
        \item $f(x) = \sin(x), k = 1, p = U[-\frac{\pi}{2}, \frac{\pi}{2}]$
        \item $(x) = e^{x/2}, k = 1, p = \mathcal{N}(0, 1)$
        \item $f(x) = \sqrt{1 - x}, k = 1, p = B(1, \frac{3}{2})$ (бета-распределение)
    \end{enumerate}
    
    \noindent
    \underline{\it Указание}: В последнем пункте воспользуйтесь табличными значениями гамма-функции.
\end{esProblem}

\begin{esProblem}
    Пусть $x \in \mathbb{R}^3$, и значения признаков равномерно распределены по шару радиуса $R$ c центром в нуле. Пусть $\mathbb{E}[y|x] = \|x\|_2$. Найдите смещение константного алгоритма $\mu(X)(x) = C = \text{const}$. При каком значении $C$ достигается минимум смещения?
\end{esProblem}


\begin{esProblem}
    Количество мёда в горшках, которое Винни-Пух хочет съесть в гостях, можно представить в виде $y_i = 5 + \varepsilon_i$, где $\varepsilon_i$ имеет равномерное распределение $\varepsilon_i \sim \mathcal{U}[-2;2]$, а $i$ — номер визита в гости.
    
    При прогнозировании аппетита Винни-Пуха Кролик абсолютно игнорирует все его прошлые визиты и просто подкидывает правильную монетку восемь раз. При каждом выпадении орла, Кролик ставит на стол очередной горшок мёда из глубоких запасов.
    
    Постройте разложение квадратичной функции потерь прогноза Кролика на компоненты разброса, смещения и шума.
\end{esProblem}


\begin{esProblem}
    (\textbf{*}) На семинаре выводилось разложение ошибки для одномерной линейной регрессии $\mu(X)(x) = w(X)x$. Вспомним модель порождения данных, которую мы использовали. Единственный признак генерировался из нормального распределения $x \sim \mathcal{N}(0, \sigma_1^2)$, а целевая переменная $y = f(x) + \varepsilon, \varepsilon \sim \mathcal{N}(0, \sigma_2^2)$. Рассмотрим теперь обучение модели с $L_2$-регуляризацией: 
    $$
    \sum_{i=1}^{\ell} (y_i - k x_i)^2 + \lambda k^2 \rightarrow \displaystyle\min_k
    $$
    Как изменится шумовая компонента при использовании модели с регуляризацией? Найдите смещение и разброс модели для линейной $f(x) = ax$ и произвольной четной $f(x)$. Пронализируйте результаты при $\lambda \rightarrow \infty$.
\end{esProblem}

\begin{esProblem}
    (\textbf{*}) Предположим, что объекты описываются двумя независимыми признаками: $x=(x_1, x_2) \in \mathbb{R}^2$, каждый из которых имеет распределение Бернулли с параметром $p=\frac{1}{2}$. Пусть целевая переменная задана как $\mathbb{E}[y|x] = x_1 x_2$ и обучающая выборка $X$ состоит из двух объектов. Будем строить решающее дерево по следующим правилам:
    
    \begin{enumerate}
        \item Разбиение объектов в вершине продолжается, пока они отличаются значением хотя бы одного признака.
        \item Критерием информативности является дисперсия целевой переменной.
        \item В случае равенства функционалов качества предпочтение отдается разбиению по первому признаку.
    \end{enumerate}
    
    \noindent
    Найдите смещение и разброс такого решающего дерева.
\end{esProblem}

\begin{esProblem}
    (\textbf{**}) Рассмотрим пространство многочленов одной переменной степени не выше $d$: $p(x) = p_0 + p_1 x + ... + p_d x^d$. Пусть многочлены выступают в качестве объектов, а коэффициенты будут их признаками, распределенными нормально: $p_i \sim \mathcal{N}(0, 1)$. Допустим, что целевая переменная определена как $\mathbb{E}[y|p] = p(x_0)$, где $x_0$ - некоторое фиксированное (но нам неизвестное) число. Пусть обучающая выборка состоит из $d$ многочленов: $p^1, ..., p^d$. Предложите алгоритм, минимизирующий сумму смещения и разброса.
\end{esProblem}

\end{document}
