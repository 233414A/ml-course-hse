\documentclass[12pt,fleqn]{article}
\usepackage{../../vkCourseML}
\usepackage{lipsum}
\usepackage{indentfirst}
\title{Машинное обучение, ФКН ВШЭ\\Домашнее задание №8}
\author{}
\date{}
\theorembodyfont{\rmfamily}
\newtheorem{esProblem}{Задача}

\begin{document}
\maketitle

\begin{esProblem}
    В алгоритме K-Means оптимизируется внутрикластерное расстояние:
    \[
        \sum_{k = 1}^{K} \sum_{i = 1}^{\ell}
            [a(x_i) = k]
            \| x_i - c_k \|^2.
    \]
    Покажите, что
    \begin{itemize}
        \item при фиксированных центрах~$c_k$ каждый объект оптимально приписывать к кластеру,
            центр которого является ближайшим;
        \item при фиксированном распределении объектов по кластерам~$a(x_i)$
            оптимальное значения для центра вычисляется как средняя точка в кластере.
    \end{itemize}
\end{esProblem}

\begin{esProblem}
    Покажите, что алгоритм K-Means сходится за конечное число итераций,
    причём число шагов не превосходит~$K^\ell$.
\end{esProblem}

\end{document}
