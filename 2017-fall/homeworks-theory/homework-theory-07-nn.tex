\documentclass[12pt,fleqn]{article}
\usepackage{vkCourseML}
\usepackage{lipsum}
\usepackage{indentfirst}
\title{Машинное обучение, ФКН ВШЭ\\Домашнее задание №7}
\author{}
\date{}
\theorembodyfont{\rmfamily}
\newtheorem{esProblem}{Задача}

\begin{document}
    \maketitle
    
    \begin{esProblem}
        На лекции и семинаре был рассмотрен метод обратного распространения ошибки в общем случае. Рассмотрим частный случай с полносвязным слоем, у которого $d_{in}$ входных и $d_{out}$ выходных нейронов. В качестве функции активации используем сигмоидальную: $g(t) = \frac{1}{1+\exp(-t)}$. 

        В полносвязном слое $i$-й выходной нейрон для слоя $l$ можно выразить следующим образом: 
        $$x_i^l = g \left( \sum_{j=1}^{d_{in}} w_{ij}^{l} x_{j}^{l-1} + b_i^l \right)$$.

        Требуется вычислить производную функции потерь $L(z, y)$ по весу $w_{ij}^{l}$ полносвязного слоя $l$: 
        $$\frac{\partial L}{\partial w_{ij}^{l}}$$

        Выражение может включать в себя величины, посчитанные во время прямого прохода по нейронной сети, и величины, полученные со следующих (по порядку прямого прохода) слоёв во время обратного прохода. Функция потерь дифференцируема по выходам сети $z$.
    \end{esProblem}


    \begin{esProblem}
        Рассмотрим вместо полносвязного слоя из задачи 1 свёрточный слой. Пусть на вход поступает изображение размера $H \times W$, свёрточный слой имеет размер $k_1 \times k_2$. Тогда применение свёрточного слоя можно выразить следующим образом:

        $$x_{ij}^l = g \left( \sum_{m=0}^{k_1-1} \sum_{n=0}^{k_2-1} w_{mn}^{l} x_{i+m,j+n}^{l-1} \right)$$

        Требуется вычислить производную функции потерь $L(z, y)$ по весу $w_{mn}^{l}$ свёрточного слоя $l$: 
        $$\frac{\partial L}{\partial w_{mn}^{l}}$$

        Выражение может включать в себя величины, посчитанные во время прямого прохода по нейронной сети, и величины, полученные со следующих (по порядку прямого прохода) слоёв во время обратного прохода. Функция потерь дифференцируема по выходам сети $z$.
    \end{esProblem}
    

    \begin{esProblem}
        С ростом количества слоёв нейронные сети могут выделять всё более сложные структуры в исходном пространстве признаков. Однако обучение глубоких нейронных сетей с помощью градиентных методов оптимизации вызывает некоторые сложности. Одной из таких проблем является проблема затухающих градиентов, когда градиенты для весов первых слоёв оказываются близкими к нулю, из-за чего первые слои обучаются медленнее последних. Подумайте и ответьте, почему так происходит. Предлагается рассматриваться сигмоиду и гиперборлический тангенс в качестве функции активации во всей сети. 
    \end{esProblem}


    \begin{esProblem}
        Расмотрим нелинейной преобразовае ReLU: 
        $$g(x) = \begin{cases} x, x \ge 0 \\ 0, x < 0 \end{cases}$$

        Несмотря на свою простоту, оно позволяет сети выучивать сложную структуру и при этом легко вычисляется. Предлагается убедиться в нелинейности, подобрав коэффициенты сети для классификации с 2 входами и 1 выходом так, чтобы она выдавала ответы такие же, как функция XOR (нули на наборах из двух нулей и двух единиц, единицу на наборах с одной единицы). Также покажите, что для такой зависимости нельзя построить линейный классификатор, не допускающий ни одной ошибки.

        Сеть должна иметь один скрытый слой из двух нейронов с активацией ReLU и одним выходным нейроном без нелинейного преобразования. Класс выдаваемый сетью определяется с помощью некоторого порога (например, 0.5, что эквивалентно при использовании сигмоидальной фукнции выбору класса с максимальной вероятностью). Свободных членов в сети нет. Достаточно найти любое подходящее решение.
    \end{esProblem}

\end{document}
