\documentclass[12pt,fleqn]{article}

\usepackage{vkCourseML}

\usepackage{lipsum}
\usepackage{indentfirst}
\title{Машинное обучение, ФКН ВШЭ\\Матрично-вектороное дифференцирование}
\author{}
\date{}
\theorembodyfont{\rmfamily}
\newtheorem{esProblem}{Задача}
\begin{document}

\maketitle

\begin{esProblem}
    Найдите производную по матрице $A \in \mathbb{R}^{n\times n}$

    \begin{equation*}
        \frac{\partial}{\partial A} \log \det A.
    \end{equation*}
\end{esProblem}

\begin{esProblem}
    Найдите производную по вектору $a \in \mathbb{R}^{n}$

    \begin{equation*}
        \frac{\partial}{\partial a} \left(
            a^T \exp(a a^T)a
        \right),
    \end{equation*}
    где $\exp(B)$~--- \href{https://en.wikipedia.org/wiki/Matrix_exponential}{матричная экспонента},
    $B \in \mathbb{R}^{n \times n}$.
    Матричной экспонентой обозначают ряд
    \begin{equation*}
        I_n + \frac{B}{1!} + \frac{B^2}{2!} + \frac{B^3}{3!} + \frac{B^4}{4!} + \ldots = \sum_{k=0}^\infty \frac{B^k}{k!} .
    \end{equation*}
\end{esProblem}

\begin{esProblem}
    Пусть $A \in \mathbb{R}^{m \times n}, b \in \mathbb{R}^m$. Найдите производную по вектору $x \in \mathbb{R}^n$
    \[
    \frac{\partial}{\partial x} \sin \|Ax + b\|_2
    \]
\end{esProblem}

\begin{esProblem}
    Рассмотрим симметричную матрицу $A \in \mathbb{R}^{n \times n}$ и ее спектральное разложение $A = Q \Lambda Q^T$. Пусть $\lambda \in \mathbb{R}^n$ - это диагональ матрицы $\Lambda$ (то есть вектор, составленный из собственных значений $A$). Найдите:
    
    \begin{enumerate}
        \item $\displaystyle\frac{\partial}{\partial \lambda} \Tr(A)$
        \item $\displaystyle\frac{\partial}{\partial Q} \Tr(A)$
    \end{enumerate}
\end{esProblem}

\begin{esProblem}
    Рассмотрим задачу обучения линейной регрессии с функцией ошибки Log-Cosh:
    \[
    Q(w) = \frac{1}{\ell} \sum_{i=1}^{\ell} \log (\cosh (w^T x_i - y_i))
    \]
    Выпишите формулу для градиента $\nabla_w Q(w)$. Запишите ее в матричном виде, используя матрицу объекты-признаки $X$ и вектор целевых переменных $y$.
\end{esProblem}

\end{document}
