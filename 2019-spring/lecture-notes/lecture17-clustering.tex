\documentclass[12pt,fleqn]{article}
\usepackage{vkCourseML}
\hypersetup{unicode=true}
%\usepackage[a4paper]{geometry}
\usepackage[hyphenbreaks]{breakurl}

\interfootnotelinepenalty=10000

\begin{document}
\title{Лекция 17\\Спектральная кластеризация}
\author{Затехано\,@TmKarter\,(Анищенко И.И.)\\ФКН ВШЭ}
\maketitle

Методы кластеризации, которые были рассмотренны ранее на лекциях:

\begin{itemize}
	\item K-means~--- метрический метод
	\item DBSCAN~--- density-based метод (алгоритм группирует вместе точки, которые тесно расположены)
\end{itemize}

В рамках этой лекции был разобран один из графовых методов кластеризации~--- {\bf Спектральная кластеризация}. Основное полезное свойство метода заключается в формировании новых признаков для задачи кластеризации на исходных данных.

\section{Построение графа}
Так как спектральная кластеризация относится к графовым методам, то для начала по имеющимся данным нам необходимо построить граф. Граф по определению это:~$G=(V,E)$, где $V=X=\{x_1,...,x_{\ell}\}$~--- вершины нашего графа в рамках нашей задачи, а ребра $(E)$ между ними можно будет задать несколькими способами:

\begin{itemize}
	\item полный граф, все рёбра с весами~$w_{ij} = \large\exp(-\frac{||x_i - x_j||^2}{2\sigma^2})$~
	\item $KNN$-граф (вершину $x_i$ соединим с $k$ соседями)
	\item $\varepsilon$-граф (соединяем вершины, между которыми расстояние меньше заданной величины:~$\rho(x_i,x_j)<\varepsilon$~)
\end{itemize}

Далее по полученному графу можно вывести его {\bf матрицу смежности} - $W$, где в ячейке $W_{ij}$ записан $0$, если вершины $x_i$ и $x_j$ не соединены, или записан его вес, если ребро между ними есть.\\
Так же запишем {\bf матрицу степеней} вершин: $D=diag(d_1,...,d_{\ell})$, где $d_i = \sum^{\ell}_{j=1}w_{ij}$ - сумма весов рёбер, соседних с вершиной $i$.\\
Через эти две матрицы мы можем получить {\bf лапласиан графа}: $L = D - W$, эта матрица нам понадобится дальше.

\section{Лапласиан графа и его свойства}
Рассмотрим свойства, которыми обладает лапласиан графа:
\begin{itemize}
	\item $f \in R^{\ell} \to f^TLf=\frac{1}{2}\sum^{\ell}_{i,j=1}w_{ij}(f_i-f_j)^2$
	\item $L$ - симметричная и неотрицательно определенная (неотрицательная определенность следует из свойства выше: $w_{ij} \geq 0$ - по определению используемых нами весов и $(f_i - f_j)^2 \geq 0$)
	\item третье свойство выведем в виде теоремы, которую далее докажем
\end{itemize}

\begin{vkTheorem}
	 
	\begin{enumerate}
		\item Собственное значение $\lambda = 0$ для матрицы $L$ имеет кратность, равную числу компонент связности графа
		\item $A_1,..,A_k$ - компоненты связности; $f_1=([x_i \in A_1])^{\ell}_{i=1},...,f_k$ - собственные вектора (это индикаторные векторы, показывающие к какой компоненте связности принадлежат вершины $x_i$), соответствующие $\lambda = 0$
	\end{enumerate}
\end{vkTheorem}

\begin{vkProof}
	Рассмотрим случай с числом компонент связности $k = 1$ (граф связный):
	разберемся, почему $\lambda = 0$ - собственное значение матрицы $L$: рассмотрим произведение $f^TLf$ с вектором $f = (1,...,1)$. Получим $f^TLf = \frac{1}{2}\sum^{\ell}_{i,j=1}w_{ij}(1 - 1)^2=0=\lambda f$\\
	Далее предположим, что для $\lambda = 0$ мы имеем какой-нибудь еще соответствующий собственный вектор $f'$, который от предыдущего отличается тем, что имеет среди своих компонент различные значения (т.е. не константный, как предыдущий вектор). Тогда если $f' \neq const \to \exists p,q: f'_p\neq f'_q$ - т.е. в нашем новом векторе есть пара различных значений.\\
	Далее вспомним, что данный граф имеет 1 компоненту связности, а значит между точками $p$ и $q$ будет иметься некий путь: $\exists p\to i_1 \to i_2 \to ... \to q$. Рассмотрим ребро $(p,i_1)$ - оно есть в нашем графе, значит $w_{p,i_1} > 0$ и для того, чтоб слагаемое с этим ребром $w_{i,i_1}(f'_p - f'_{i_1})^2$ давало $0$ в итоговую сумму $f'^TLf'$ (т.к. если $f'$ - соб. вектор, то $f'^TLf' = f'^T * 0 = 0$) нам нужно, чтобы $f'_p = f'_{i_1}$. Аналогична ситуация и с ребром $(i_1,i_2)$. И двигаясь дальше к вершине $q$ мы получим, что $f'_p=f'_q$ - тем самым получив противоречие по значениям $f_p$ и $f_q$, из которого можем вывести следующее:~$f'$ не будет собственным вектором для $\lambda = 0$.\\
	Отсюда получаем, что собственному значению $\lambda$ будет соответствовать только один собственный вектор: $f = const$, что и даем нам кратность для этого соб.зн. $1 = k$ - числу компонент связности. И сам вектор представляет из себя набор индикаторных значений принадлежности к классу, что и рассматривалось во $2$ пункте\\
	
	Случай с $k > 1$: в получившейся матрице $L$ сможем упорядочить вершины так, чтобы получить блочно диагональную матрицу:
	$$L' = 
	\begin{pmatrix}
	L_1 & 0 & \dots & 0 \\
	0 & L_2 & \dots & 0 \\
	0 & 0 & \ddots & 0 \\ 
	0 & \dots & 0 & L_k
	\end{pmatrix}$$
	где $L_1,L_2,...,L_k$ - лапласианы компонент связности. Дальше вспомним, что спектр матрицы $L$ (её мн-во собственных значений) в блочно диагональном виде матрицы будет характеризоваться, как объединение спектров маленьких матриц $L_1,L_2,...,L_k$. У каждой этой матрицы в отдельности есть нулевое собственное значение с кратностью 1, и ему соотвествует единственный собственный вектор $f=const$.\\
	Тогда при объединении этих спектров у $\lambda = 0$ кратность будет $k$, а собственными векторами к этому собственному значению будут вектора вида: $f_1=([x_i \in A_1])^{\ell}_{i=1},...,f_k$ (в векторе $i$ компонента $j$ будет принимать значение $0$, если вершина $x_j$ не лежит в компоненте связности $i$, и будет принимать 1 в противном случае)
\end{vkProof}

Далее перейдем к гипотезе, которая не доказывается строго, но подтверждается экспериментально

\begin{vkHypothesis}
	У собственных векторов $f_i$, соответствующие маленьким собственным значениям, $f_{ij}\approx f_{ik} \Leftrightarrow x_j$ и $x_k$ близки в графе (т.е. между ними расстояние в графе небольшое).
\end{vkHypothesis}
Под маленькими собственными значениями мы подразумеваем набор из наименьших собственных значений матрицы $L$

\section{Алгоритм спектральной кластеризации}
На основе гипотезы выше можем вывести алгоритм спектральной кластеризации:
\begin{enumerate}
	\item Строим $L = D - W$; сложность этого шага $O(\ell^2)$
	\item Из лапласиана $L$ находим $u_1,...,u_k$ - нормированную систему из $k$ собственных векторов, соответствующую минимальным собственным значениям; сложность $O(\ell^3)$
	\item Составляем матрицу $U = (u_1|u_2|...|u_k)$ размера $\ell \times k$. Суть этой матрицы в следующем: по строчкам расположено описание $\ell$ объектов, а каждый вектор $f_i$ записан в столбец. Тогда если мы рассмотрим признаки похожих объектов $x_i$ и $x_j$ (близки друг к другу в графе), то в каждая их пара будет приблизительно равна, тогда и $f_{ij} \approx f_{ik}$
	\item Кластеризуем $U$ с помощью $K-means$
\end{enumerate}
И после получим нужную нам кластеризацию.

\section{Внешние метрики}
Теперь что-то стоит сказать про метрики, которые оценивают качество проводимой нами кластеризации. Для оценки качества будем пользоваться нашей моделью $a(x)$ разметкой с истинными ответами ($(y_1,...,y_{\ell})$~- $golden~standard$). Фактически тут мы будем проводить supervised проверкой качества для unsupervised моделей, так как это единственный способ измерить качество моделей классификации.

\begin{figure}[t]
	\centering
	\includegraphics[width=\textwidth]{pic.png}
	\caption{Требования к правильному сравнению результатов кластеризации во внешних метриках}
	\label{fig:terms}
\end{figure}

\subsection{Требования к внешней метрике}
Так как имеющаяся у нас разметка не совсем является целевой переменной (к прим. в истинной разметке нам даны классы как метки к статьям: наука, политика, спорт. А наша модель выдает численные метки для разделения на классы) и это отображение истинных меток к нашей модели может быть подобранно не единственным способом, то было придуманно большое множество внешних методов для получения оценки качества. Но у них есть свои проблемы, которые заключаются в правильном сравнении результатов кластеризации. Лучшее разбиение то, где все объекты с одинаковой истинной меткой попали в один кластер или в разные кластеры? Лучше когда объекты с двумя видами меток попали в один кластер или в разные? Для детерминированности поведения в этих различных ситуациях были введены некоторые требования к внешней метрике (схематично они показаны на рис.1). Можно выделить 4 свойства, которыми она должна обладать:

\begin{enumerate}
	\item Гомогенность. Если мы имеем кластер с несколькими видами меток, то для улучшения разбиения мы хотим потребовать, чтобы эти объекты были разделены по разным кластерам в зависимости от значения меток. 
	\item Полнота. Если мы имеем несколько кластеров с одним и тем же видом истинной метки, то для улучшения качества разбиения мы хотим \grqq слить\grqq~эти объекты в один кластер.
	\item Ray Bay. Пусть мы имеем два кластера и один из них \grqq мусорный\grqq~(в нем много объектов с разными метками). А первый кластер содержит много объектов одного вида метки и несколько \grqq чужих\grqq~(объекты с другой меткой). Тогда наше разбиение будет лучше, если лишние объекты из первого кластера окажутся в нашем \grqq мусорном\grqq~кластере.
	\item Не имеет конкретного названия, но утверждает следующее. Пусть мы имеем большой кластер с объектами одного вида истинной метки и много других кластеров с другими видами меток (среди них есть кластеры с одинаковыми видами меток, но наша модель их выделила как отдельные группы). Если ценой разделения большого первого кластера на два отдельных (\grqq отщипнули\grqq~от большого кластера один объект) мы добьемся слияния нужных нам кластеров с другими метками (то, что нам нужно хорошо показано на картинке), то наша кластеризация станет лучше, чем была первоначально.
\end{enumerate}

\subsection{Метрика BCubed}
Если мы решим подыскать себе метрику со всеми выполняющимися 4 требованиями, то нам может подойти метрика {\bf BCubed}. Для её понимания введём следующие обозначения: $L(x)$ - golden standard (истинная метка объекта), $C(x)$ - номер кластера объекта. Так же введём дополнительную меру 
$$Correctness(x,x')=
	\begin{cases}
	1;~C(x)=C(x')~\text{и}~L(x)=L(x') \\
	0;~\text{иначе}
	\end{cases}$$ 
И на основе всего введём две составляющих этой метрики:
\begin{equation*}
	Precision~BCubed = Avg_x(Avg_{(C(x')=C(x))}~Correctness(x,x'))
\end{equation*}
\begin{equation*}
	Recall~BCubed = Avg_x(Avg_{(L(x')=L(x))}~Correctness(x,x'))
\end{equation*}
Далее от этих метрик можно посчитать F-меру, которая и будет удовлетворять всем 4 требованиям к внешним метрикам.

\end{document}