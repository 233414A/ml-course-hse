\documentclass[12pt,fleqn]{article}
\usepackage{vkCourseML}
\usepackage{gensymb}
\hypersetup{unicode=true}
%\usepackage[a4paper]{geometry}
\usepackage[hyphenbreaks]{breakurl}

\interfootnotelinepenalty=10000

\begin{document}
\title{Лекция 16\\Одноклассовые методы и обнаружение аномалий}
\author{Е.\,А.\,Соколов\\ФКН ВШЭ}
\maketitle

В задачах кластеризации, о которых шла речь ранее, требуется разделить выборку
на группы так, чтобы внутри каждой группы объекты были похожи друг на друга.
Теперь мы изучим немного другую постановку~--- поиск аномалий.
В ней даётся выборка~<<нормальных>> объектов, и требуется построить некоторую модель,
описывающую данную выборку.
Далее для новых объектов требуется определять, принадлежат ли они тому же распределению,
что и эта выборка, или же являются выбросами или аномалиями.
Такие методы применяются, например, в задачах обнаружения мошеннического поведения
или раннего обнаружения неполадок оборудования.

\section{Несбалансированная классификация}

В некоторых задачах примеры аномалий могут быть даны, но в небольших объёмах~---
например, при анализе данных систем самолёта может быть
известно несколько аномальных ситуаций из прошлого.
Такую задачу можно рассматривать как классификацию с несбалансированными классами.
При решении обычными методами классификатору оказаться выгоднее относить все объекты к одному классу,
поэтому имеет смысл модифицировать процедуру обучения.

Самые простые методы борьбы с несбалансированностью~--- \emph{undersampling} и \emph{oversampling}.
Первый из них удаляет случайные объекты доминирующего класса до тех пор,
пока соотношение классов не станет приемлемым;
второй дублирует случайные объекты минорного класса.
Оптимальное число объектов для удаления или дублирования следует подбирать с помощью кросс-валидации.
Отметим, что данные методы применяются лишь к обучающей выборке,
а контрольная выборка остается без изменений.

Более сложный метод SMOTE~\cite{chawla02smote} заключается в дополнении минорного класса
синтетическими объектами.
Генерация нового объекта производится следующим образом.
Выбирается случайный объект~$x_1$ минорного класса, для него выделяются~$k$ ближайших
соседей из этого же класса~($k$~--- настраиваемый параметр), из этих соседей
выбирается один случайный~$x_2$.
Новый объект вычисляется как точка на отрезке между~$x_1$ и~$x_2$:
$\alpha x_1 + (1 - \alpha) x_2$, для случайного~$\alpha \in (0, 1)$.

\section{Одноклассовая классификация}

Ниже мы будем обсуждать обнаружение точечных аномалий~--- объектов,
которые существенно отличаются от заданной выборки.
При этом выделяют и другие типы.
Так, контекстными аномалиями называют наблюдения, отличающиеся
от наблюдений, близких по некоторому параметру.
Например, температура $-10\degree$ является нормальной в январе,
но аномальной в июне.

\subsection{Статистические методы}

В статистических методах предлагается восстановить плотность выборки~$p(x)$,
и затем определять аномальность объекта на основе того,
насколько вероятно его получить из данной плотности.
Например, это можно делать через отклонение от среднего~$[\rho(x, \mu) > d]$~(порог
может подбираться, если известно некоторое количество примеров аномалий),
сравнение значения плотности с порогом~$[p(x) < d]$
или с помощью статистических тестов.
Существует два подхода к восстановлению плотности: параметрический
и непараметрический.

\subsubsection{Непараметрический подход}

Начнём с одномерных величин.
Согласно одному из определений неотрицательная функция~$p(x)$
является плотностью распределения случайной величины~$\xi$, если её значение в каждой точке
равно пределу
\[
    p(x)
    =
    \lim_{h \to 0}
        \frac{1}{2h}
        \PP(\xi \in [x - h, x + h]).
\]
Воспользуемся этим определением и построим эмпирическую оценку плотности:
\[
    \hat p(x)
    =
    \frac{1}{2h} \frac{1}{\ell}
    \sum_{i = 1}^{\ell}
        \left[
            |x - x_i| < h
        \right]
    =
    \frac{1}{\ell h}
    \sum_{i = 1}^{\ell}
        \frac12
        \left[
            \frac{|x - x_i|}{h} < 1
        \right],
\]
где~$h$~--- ширина окна, регулирующая гладкость эмпирической плотности.
Чем больше объектов обучающей выборки в окрестности точки, тем выше будет плотность.

В указанной оценке используется индикатор, что приводит к отсутствию гладкости.
Чтобы устранить это, заменим индикатор того, что расстояние меньше ширины окна,
на некоторую гладкую функцию~$K(z)$:
\[
    \hat p(x)
    =
    \frac{1}{\ell h}
    \sum_{i = 1}^{\ell}
        K
        \left(
            \frac{x - x_i}{h}
        \right).
\]
Здесь~$K(z)$~--- ядро~(не путайте с ядрами Мерсера!), которое должно удовлетворять
четырём требованиям:
\begin{itemize}
    \item чётность: $K(-z) = K(z)$;
    \item нормированность: $\int K(z) dz = 1$;
    \item неотрицательность: $K(z) \geq 0$;
    \item невозрастание при~$z > 0$.
\end{itemize}
Примером может служить гауссово ядро~$K(z) = (2 \pi)^{-1/2} \exp(-0.5 z^2)$.

Оценку плотности легко обобщить на многомерный случай, заменив разность~$|x - x_i|$
на некоторую метрику~$\rho(x, x_i)$:
\begin{equation}
\label{eq:nonparametric}
    \hat p(x)
    =
    \frac{1}{\ell V(h)}
    \sum_{i = 1}^{\ell}
        K
        \left(
            \frac{\rho(x, x_i)}{h}
        \right),
\end{equation}
где~$V(h) = \int K \left( \frac{\rho(x, x_i)}{h} \right) dx$~--- нормировочная константа.
Следует помнить, что число объектов, необходимое для качественной оценки плотности,
растёт экспоненциально по мере роста числа признаков.
Из-за этого непараметрические методы подходят только для обнаружение аномалий
в маломерных пространствах.

\subsubsection{Параметрический подход}

Параметрический подход состоит в приближении плотности с помощью распределения~$p(x \cond \theta)$
из некоторого семейства~$\{p(x \cond \theta) \cond \theta \in \Theta\}$
с помощью метода максимального правдоподобия:
\[
    \sum_{i = 1}^{\ell}
        \log p(x_i \cond \theta)
    \to
    \max_\theta
\]
В качестве распределений могут выступать, например, нормальные или смеси нормальных.
В пространствах большой размерности может иметь смысл наивное байесовское предположение,
о котором пойдёт речь на семинарах.

\subsection{Метрические методы}

Метрический подход основан на выделении объектов, которые расположены от других
существенно дальше, чем объекты в среднем удалены друг от друга.
А именно, объект~$x$ объявляется аномальным, если~$p$ или меньше процентов объектов
имеют до него расстояние меньше~$\eps$:
\[
    \frac{1}{\ell}
    \sum_{i = 1}^{\ell}
        [\rho(x, x_i) < \eps]
    \leq
    p.
\]
Пороги~$p$ и~$\eps$ являются параметрами, которые должны настраиваться
по известным примерам аномалий или исходя из априорных предположений.

\subsection{Одноклассовый метод опорных векторов}

Заметим, что похожим образом можно применять любую модель для обнаружения аномалий~---
достаточно обучить её так, чтобы прогнозы для объектов из обучения были близки к нулю
или, наоборот, как можно сильнее отделены от нуля.

Выше мы пытались описать данные с помощью распределения или использовать метрику, чтобы оценить аномальность объекта.
Далее мы разберём подход на основе моделей.
Действительно, можно взять любую модель машинного обучения и настроить её так,
чтобы на нормальных объектах она принимала близкие к нулю или, например, положительные значения.
Тогда можно будет считать, что если на новом объекте прогноз сильно отличается от прогнозов на обучающей выборке,
то этот объект скорее аномальный.
Мы поговорим о двух методах: на основе SVM и на основе решающих деревьев.

Для обнаружения аномалий, по сути, необходимо построить некоторую функцию~$a(x)$,
которая принимает значение~$1$ на области как можно меньшего объёма,
содержащей как можно больше объектов выборки; во всех остальных точках она
должна иметь значение~$0$.
Такая функция будет компактно описывать обучающую выборку,
и можно рассчитывать, что на аномальных объектах она будет отрицательной.

Будем строить линейную функцию~$a(x) = \sign \langle w, x \rangle$, и потребуем,
чтобы она отделяла выборку от начала координат с максимальным отступом.
Соответствующая оптимизационная задача будет иметь вид~\cite{scholkopf99oneclass}
\[
    \left\{
        \begin{aligned}
            & \frac{1}{2} \|w\|^2
            +
            \frac{1}{\nu \ell} \sum_{i = 1}^{\ell} \xi_i
            -
            \rho
            \to \min_{w, \xi, \rho} \\
            & \langle w, x_i \rangle
            \geq
            \rho - \xi_i,
            \quad i = 1, \dots, \ell, \\
            & \xi_i \geq 0, \quad i = 1, \dots, \ell.
        \end{aligned}
    \right.
\]

Здесь гиперпараметр~$\nu$ отвечает за корректность на обучающей выборке~---
можно показать, что он является верхней границей на число аномалий~(объектов выборки,
на которых~$a(x) = -1$).
Решающее правило будет иметь вид
\[
    a(x)
    =
    \sign\left(
        \langle w, x \rangle
        -
        \rho
    \right),
\]
где ответ~$-1$ будет соответствовать выбросу.
Получается, что мы ищем гиперплоскость так, что:
\begin{itemize}
    \item она отделяет как можно больше объектов выборки от нуля~(чем меньше~$\nu$, тем больше объектов мы будем отделять)~---
        за это отвечает слагаемое~$\frac{1}{\nu \ell} \sum_{i = 1}^{\ell} \xi_i$ в функционале;
    \item она имеет большой отступ~$\frac{1}{\|w\|^2}$;
    \item она при этом как можно сильнее отдалена от нуля~(то есть~$\rho$ как можно большее значение).
\end{itemize}

Для данной задачи можно выписать двойственную и сделать ядровой переход в ней:
\[
    \left\{
        \begin{aligned}
            & \frac{1}{2} \sum_{i, j = 1}^{\ell}
                \lambda_i \lambda_j K(x_i, x_j)
            \to \min_{\lambda} \\
            & 0 \leq \lambda_i \leq \frac{1}{\nu \ell},
            \quad i = 1, \dots, \ell, \\
            & \sum_{i = 1}^{\ell} \lambda_i = 1.
        \end{aligned}
    \right.
\]
Модель при этом будет иметь вид
\[
    a(x)
    =
    \sign\left(
        \sum_{i = 1}^{\ell}
            \lambda_i K(x, x_i)
        -
        \rho
    \right).
\]
Заметим, что при использовании гауссова ядра данная модель будет очень похожа
на метод, который строит непараметрическую оценку плотности~\eqref{eq:nonparametric} с гауссовым ядром
и сравнивает её значение с порогом~$\rho$.

При использовании подходящих ядер можно действительно получить функцию, которая
точно описывает обучающую выборку в исходном пространстве.
Также можно показать, что объекты из того же распределения, из которого сгенерирована
обучающая выборка, будут с не очень большой вероятностью попадать в область
с отрицательным значением~$a(x)$.


\subsection{Isolation forest}

Ранее мы обсуждали, что случайный лес вводит функцию расстояния~--- чем чаще два объекта попадают
в один лист, тем более похожими их можно считать.
Похожий подход можно использовать и для обнаружения аномалий.
Метод, который мы разберём, называют изоляционным лесом~(Isolation forest)~\cite{liu08iforest}.

На этапе обучения будем строить лес, состоящий из~$N$ деревьев.
Каждое дерево будем строить стандартным жадным алгоритмом,
но при этом признак и порог будем выбирать случайно.
Строить дерево будем до тех пор, пока в вершине не окажется ровно один объект,
либо пока не будет достигнута максимальная высота.
Высоту дерева можно ограничить величиной~$\log_2 \ell$.

Метод основан на предположении о том, что чем сильнее объект отличается от большинства,
тем быстрее он будет отделён от основной выборки с помощью случайных разбиений.
Соответственно, выбросами будем считать те объекты, которые оказались на небольшой глубине.

Чтобы вычислить оценку аномальности объекта~$x$, найдём расстояние от соответствующего ему листа
до корня в каждом дереве.
Если лист, в котором оказался объект, содержит только его, то в качестве оценки~$h_n(x)$
от данного~$n$-го дерева будем брать саму глубину~$k$;
если же в листе оказалось~$m$ объектов, то в качестве оценки возьмём величину~$h_n(x) = k + c(m)$.
Здесь~$c(m)$~--- средняя длина пути от корня до листа в бинарном дереве поиска, которая вычисляется по формуле
\[
    c(m)
    =
    2H(m - 1) - 2\frac{m - 1}{m},
\]
а~$H(i) \approx \ln(i) + 0.5772156649$~---~$i$-е гармоническое число.
Оценку аномальности вычислим на основе средней глубины, нормированной на среднюю длину пути в дереве,
построенном на выборке размера~$\ell$:
\[
    a(x)
    =
    2^{-\frac{
            \frac{1}{N} \sum_{n = 1}^{N} h_n(x)
        }{
            c(\ell)
        }
    }.
\]
Для ускорения работы можно строить каждое дерево на подвыборке размера~$s$;
в этом случае во всех формулах выше нужно заменить~$\ell$ на~$s$.


\begin{thebibliography}{1}
\bibitem{chawla02smote}
    \emph{Chawla N., Bowyer K., Hall L., Kegelmeyer W.} (2002).
    SMOTE: Synthetic Minority Over-sampling Technique.~//
    Journal of Artificial Intelligence Research, Vol. 16, Pp. 321–357.

\bibitem{scholkopf99oneclass}
    \emph{Sch\"{o}lkopf, Bernhard and Williamson, Robert and Smola, Alex and Shawe-Taylor, John and Platt, John} (1999).
    Support Vector Method for Novelty Detection.~//
    NIPS'99.

\bibitem{liu08iforest}
    \emph{Liu, Fei Tony, Ting, Kai Ming and Zhou, Zhi-Hua} (2008).
    Isolation forest.~//
    Data Mining, 2008. ICDM‘08.
\end{thebibliography}

\end{document}
