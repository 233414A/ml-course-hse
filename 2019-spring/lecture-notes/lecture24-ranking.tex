\documentclass[12pt,fleqn]{article}
\usepackage{vkCourseML}
\usepackage{gensymb}
\hypersetup{unicode=true}
%\usepackage[a4paper]{geometry}
\usepackage[hyphenbreaks]{breakurl}

\interfootnotelinepenalty=10000

\begin{document}
\title{Лекция 24\\Обучение ранжированию}
\author{Е.\,А.\,Соколов\\ФКН ВШЭ}
\maketitle

В задаче ранжирования требуется построить модель, которая правильным образом
задаёт порядок на объектах.
Сюда относится, например, задача поискового ранжирования, где документы сортируются
по убыванию релевантности~(соответствия) запросу.
Также к ранжированию может быть сведена задача построения рекомендаций~(отсортировать товары
для данного пользователя),
исправления опечаток (выбрать наиболее подходящие варианты исправления) и т.д.
Ниже мы увидим, что задачу ранжирования можно свести к классификации или регрессии~---
но также рассмотрим и более сложные подходы, которые оптимизируют непосредственно
качество сортировки.

\section{Постановка задачи}
Пусть задано пространство объектов~$\XX$~(например,~$\RR^d$)
и некоторая обучающая выборка~$X = \{x_1, \dots, x_\ell\} \subset \XX$.
Также пусть дан некоторый порядок на объектах обучающей выборке~---
то есть набор таких пар~$(i, j) \in R \subset \{1, \dots, \ell\}^2$, что первый объект из пары
должен стоять после сортировки моделью выше второго объекта.
Множество пар~$R$ заменяет собой целевую переменную.

Например, каждый объект может представлять собой пару~$(q, d)$,
состоящую из запроса и документа.
В этом случае порядок будет задан только на таких парах~$(q_1, d_1)$ и~$(q_2, d_2)$,
которые соответствуют одному запросу~($q_1 = q_2$).
Далее мы будем рассуждать именно в терминах такого поискового ранжирования.

Требуется построить такую модель~$a: \XX \to \RR$,
что для~$(i, j) \in R$~(и только для них) выполнено~$a(x_i) < a(x_j)$.
Разумеется, на больших выборках вряд ли получится идеально выполнить это требование,
поэтому необходимо ввести метрику, определяющую качество решения задачи.

\section{Метрики качества ранжирования}

Часто в ранжировании ответы задаются не в виде пар, а, для простоты,
в виде чисел~$y_1, \dots, y_\ell$.
При этом считается, что если~$y_i < y_j$, то для модели должно быть выполнено~$a(x_i) < a(x_j)$.

Для начала рассмотрим простой случай бинарных ответов~$y_i \in \{0, 1\}$~---
грубо говоря, каждый документ либо соответствует запросу, либо не соответствует.
В этом случае можно применять любые стандартные метрики качества классификации~---
точность, полноту, F-меру, AUC-ROC и т.д.
Как правило, их вычисляют в рамках одного запроса, и затем усредняют по всем запросам из выборки.

Рассмотрим для примера точность~(precision) на одном запросе~$q$.
Поскольку нас в первую очередь интересует, какие документы оказываются
в самом верху поисковой выдачи, логично рассматривать метрику~$\text{precision}@k(q)$~---
точность, вычисленную для документов, которые модель поместила на первые~$k$ мест.
Данная метрика будет равна единице, если все~$k$ документов релевантные,
нулю, если они все нерелевантны.
При этом она никак не учитывает порядок внутри первых~$k$ позиций~--- релевантный документ
и на первой, и на~$k$-й позиции имеет одинаковый вклад.
Чуть более сложной метрикой является~AP(average precision):
\[
    \text{AP}@k(q)
    =
    \sum_{i = 1}^{k}
        \frac{
            y_{(i)}
        }{
            \sum_{j = 1}^{k} y_{(j)}
        }
        \text{precision}@i(q),
\]
где~$y_{(i)}$~--- релевантность документа на~$i$-й позиции.
В ней уже учитывается порядок, и документ на первой позиции имеет больший вес.
Значение AP, усреднённое по всем запросам, называется MAP~(mean average precision).

Если ответы являются вещественными~(например, при наличии нескольких уровней релевантности),
то можно использовать метрику~DCG~(discounted cumulative gain):
\[
    \text{DCG@k}(q)
    =
    \sum_{i = 1}^{k}
        g(y_{(i)}) d(i).
\]
Примерами конкретных функций могут служить~$g(y) = 2^{y} - 1$ и~$d(i) = \frac{1}{\log(i + 1)}$.
Чтобы значение метрики легче было интерпретировать, её можно поделить на значение DCG
при идеальном ранжировании~--- в этом случае получим метрику~nDCG~(normalized DCG):
\[
    \text{nDCG@k}(q)
    =
    \frac{
        \text{DCG@k}(q)
    }{
        \max \text{DCG@k}(q)
    }.
\]
Далее значение nDCG можно усреднить по всем запросам.

Ещё один пример метрики ранжирования~--- это pFound, предложенная в компании Яндекс.
Пусть ответы лежат на отрезке~$[0, 1]$ и отражают вероятность найти ответ в данном документе.
Зададим величину~$p_i$, соответствующую вероятности дойти до~$i$-й позиции.
Для первой позиции она равна единице, поскольку пользователь точно посмотрит
на первый документ:~$p_1 = 1$.
Вероятность дойти до~$(i+1)$-й позиции вычисляется как
\[
    p_{i + 1}
    =
    p_i
    (1 - y_{(i)})
    (1 - p_\text{out}),
\]
где~$p_\text{out}$~--- вероятность того, что пользователь
уйдёт, не узнав ответ на свой запрос.
Метрика pFound равна вероятности найти ответ среди первых~$k$ документов:
\[
    \text{pFound}@k(q)
    =
    \sum_{i = 1}^{k}
        p_i y_{(i)}.
\]
Далее она, как и другие метрики, усредняется по всем запросам.
Отметим, что pFound является~\emph{каскадной}~---
она учитывает, что пользователь просматривает поисковую выдачу
сверху вниз, и что польза документа зависит от документов выше него.

\section{Признаки в моделях ранжирования}
В задачах поискового ранжирования выделяют три типа признаков:
\begin{itemize}
    \item Запросные~--- зависят только от запроса.
        Сюда может относиться, например, популярность запроса,
        его тип~(навигационный, товарный и т.д.),
        число слов в нём.
    \item Статические~--- зависят только от документа и могут быть рассчитаны заранее.
        Сюда могут относиться популярность документа~(число ссылок на него в сети),
        его тематики, распределение слов в нём, средний word2vec-вектор и т.д.
    \item Динамические~--- зависят от запроса и документа.
        Сюда могут относиться, например, различные расстояния между запросом и документом.
\end{itemize}
Разберём несколько популярных поисковых признаков.

\paragraph{BM25}
Документ и запрос можно сравнить, например, путём подсчёта косинусного расстояния
между их TF-IDF-представлениями.
Более общим способов вычисления близости является функция Okapi BM25.
Пусть запрос~$q$ состоит из слов~$q_1, \dots, q_n$.
Тогда его сходство с документов вычисляется как
\[
    \text{BM25}(q, d)
    =
    \sum_{i = 1}^{n}
        \text{IDF}(q_i)
        \frac{
            \text{tf}(q_i, d)
            (k_1 + 1)
        }{
            \text{tf}(q_i, d)
            +
            k_1 \left(
                1 - b + b \frac{|D|}{\bar n_d}
            \right)
        },
\]
где~$\text{tf}(q_i, d)$~--- число вхождений слова~$q_i$ в документ~$d$,
$|D|$~--- число документов в выборке,
$\bar n_d$~--- средняя длина документа,
а~IDF~(inverse document frequency) может вычисляться по формуле
\[
    \text{IDF}(q_i)
    =
    \log \frac{
        |D|
    }{
        |\{d \in D \cond q_i \in d\}|
    },
\]
т.е. как доля документов, содержащих данное слово.
Величины~$b$ и~$k_1$ являются параметрами.

Метрика BM25 выводится из определённых вероятностных предположений о релевантности
документов запросам, но мы не будем останавливаться на них в данном тексте.

\paragraph{PageRank}
Алгоритм PageRank позволяет найти для каждого документа величину,
характеризующую его~<<важность>>.
Документы в сети ссылаются друг на друга, образуя граф.
Документ считается важным, если на него ссылается много документов,
которые, в свою очередь, мало на кого ссылаются.
Формально PageRank для документа~$d$ задаётся как
\begin{equation}
\label{eq:pagerank}
    \text{PR}(d)
    =
    \frac{1 - \delta}{|D|}
    +
    \delta
    \sum_{c \in D_d^\text{in}}
        \frac{
            \text{PR}(c)
        }{
            |D_c^\text{out}|
        },
\end{equation}
где~$D$~--- множество всех документов,
$D_d^\text{in}$ и~$D_d^\text{out}$~--- множества документов, от которых и к которым ведут рёбра из~$d$ соответственно.

Данная формула, по сути, отражает вероятность попасть на документ~$d$ при случайном блуждании по сети.
Согласно ней, пользователь стартует из некоторого документа и либо переходит по одной из ссылок в нём с равными вероятностями,
либо с вероятностью~$\delta$ переходит на случайную страницу из сети.

Уравнения~\eqref{eq:pagerank} можно переписать в векторном виде:
\[
    R
    =
    \frac{1 - \delta}{|D|}
    +
    \delta A R,
\]
где~$A$~--- модифицированная матрица смежности, где~$a_{ij} = \frac{1}{|D_j^\text{out}|}$,
если $j$-й документ ссылается на~$i$-й, и~$a_{ij} = 0$ в противном случае.
Через~$R$ здесь обозначен вектор~$(\text{PR}(d_1), \dots, \text{PR}(d_|D|))$.
Отсюда получаем, что вектор~$R$ можно найти путём обращения регуляризованной матрицы смежности:
\[
    R
    =
    (I - \delta A)^{-1}
    \frac{1 - \delta}{|D|}
    \vec 1.
\]
Поскольку обращать матрицу смежности может быть слишком сложно,
можно искать вектор~$R$ итерационно, инициализировав его случайным образом
и пересчитывая значения по формулам~\eqref{eq:pagerank}.
Такой подход является примером использования метода простых итераций.

\section{Методы ранжирования}
На предыдущей лекции мы договорились, что ответы задаются не в виде пар,
а в виде чисел~$y_1, \dots, y_\ell$.
При этом считается, что если~$y_i < y_j$, то для модели должно быть выполнено~$a(x_i) < a(x_j)$.

Также мы обсудили ряд метрик качества ранжирования~--- среди них MAP, nDCG и pFound.
Все они непосредственно учитывают порядок документов, и поэтому зависят от позиций, на которые
их поместила модель.
Поскольку зависимость позиций документов от параметров модели является кусочно-постоянной~(позиции ведь
принимают натуральные значения),
непосредственно обучаться с данными метриками достаточно сложно.
Поэтому многие методы ранжирования пытаются оптимизировать другие метрики,
которые являются дифференцируемыми и при этом как-то оценивают качество ранжирования.

\subsection{Поточечные методы}

В поточечном~(pointwise) подходе предлагается забыть про то, что мы решаем задачу ранжирования,
и независимо для каждого объекта~$x_i$ предсказывать ответ~$y_i$.
В зависимости от типа ответов получим задачу классификации или регрессии.

Если модель~$a(x)$, которая получится в результате такого обучения, будет идеально
восстанавливать целевую переменную, то и метрика качества ранжирования будет оптимизирована.
Если же добиться идеального восстановления целевой переменной нельзя,
то могут возникнуть проблемы~--- ведь поточечная метрика никак не учитывает порядок.
С её точки зрения необходимо как можно точнее восстановить ответы,
тогда как с точки зрения метрики ранжирования можно пожертвовать точностью предсказания
ради корректности порядка~(скажем, прогнозы~$3$ и~$4$ не будут отличаться от прогнозов
$3.9$ и $3.95$ в плане качества ранжирования друг относительно друга).

\subsection{Попарные методы}

Вспомним, что изначально мы определяли задачу ранжирования через пары объектов.
Если записывать это формально, то получим функционал ошибки
\[
    \sum_{(i, j) \in R}
        [a(x_j) - a(x_i) < 0],
\]
где~$R$~--- множество пар, для которых известен порядок.
Этот функционал не является дифференцируемым, но мы уже решали такую проблему в линейной классификации~---
надо заменить индикатор ошибки~$[z < 0]$ на его гладкую верхнюю оценку~$L(z)$:
\[
    \sum_{(i, j) \in R}
        [a(x_j) - a(x_i) < 0]
    \leq
    \sum_{(i, j) \in R}
        L\left(
            a(x_j) - a(x_i)
        \right).
\]

В качестве оценки~$L(z)$ можно брать, например, логистическую функцию~$L(x) = \log(1 + e^{-\sigma z})$ с параметром~$\sigma > 0$~---
в этом случае получим метод RankNet.
Оптимизировать данный функционал можно обычным стохастическим градиентным спуском.
Если использовать линейную модель~$a(x) = \langle w, x \rangle$, то один шаг будет иметь вид
\[
    w
    :=
    w
    +
    \eta
    \frac{
        \sigma
    }{
        1 + \exp(
            \sigma
            \langle x_j - x_i, w \rangle
        )
    }
    (x_j - x_i).
\]
Существует эмпирическое наблюдение, позволяющее перейти к оптимизации произвольной метрики ранжирования~$F$.
Оказывается, для этого надо домножить смещение на изменение метрики~$\Delta F_{ij}$, которое произойдёт при перестановке~$x_i$ и~$x_j$
местами в ранжировании:
\[
    w
    :=
    w
    +
    \eta
    \frac{
        \sigma
    }{
        1 + \exp(
            \sigma
            \langle x_j - x_i, w \rangle
        )
    }
    |\Delta F_{ij}|
    (x_j - x_i).
\]
Данный метод носит название LambdaRank.


Также по аналогии с методом опорных векторов можно вывести метод RankSVM:
\[
    \left\{
    \begin{aligned}
        &\frac{1}{2} \|w\|^2 + C \sum_{(i, j) \in R} \xi_{ij} \to \min_{w, \xi}\\
        &\langle w, x_j - x_i \rangle \geq 1 - \xi_{ij}, \quad (i, j) \in R;\\
        &\xi_{ij} \geq 0, \quad (i, j) \in R.
    \end{aligned}
    \right.
\]

Отметим, что нередко именно попарный подход оказывается лучшим при решении задачи ранжирования.

\subsection{Списочные методы}

Мы уже отмечали выше, что непосредственно оптимизировать метрику качества ранжирования вряд ли
получится из-за её дискретной структуры.
Такая проблема возникала и раньше~(например, при обучении метрик или при визуализации),
и решали мы её путём введения некоторого вероятностного распределения.
Разберём метод ListNet, который позволяет непосредственно учитывать порядок
объектов в процессе обучения.

Рассмотрим один запрос~$q$, для которого надо отранжировать~$n_q$ документов~$(d_1, \dots, d_{n_q})$.
Для этих документов известны истинные оценки релевантности~$(y_1, \dots, y_{n_q})$,
которые определяют истинное ранжирование.
Пусть также имеет некоторая модель~$a(x)$, которая выдаёт оценки~$(z_1, \dots, z_{n_q})$.
Метрика качества ранжирования~(например, nDCG) измеряет, насколько ранжирования по оценкам модели
близко к истинному ранжированию.

В постановке, которую мы только что описали, модель выдаёт одну конкретную перестановку документов для данного
запроса, и мы измеряем качество этой перестановки.
Сгладим этот процесс: предположим, что на самом деле модель выдаёт распределение на всех возможных перестановках
документов, причём вероятность конкретной перестановки~$\pi$ определяется как
\[
    P_z(\pi)
    =
    \prod_{j = 1}^{n_q}
        \frac{
            \phi(z_{\pi(j)})
        }{
            \sum_{k = j}^{n_q}
                \phi(z_{\pi(k)})
        },
\]
где~$\phi$~--- произвольная неубывающая и строго положительная функция.
Данные вероятности обладают рядом полезных свойств:
\begin{itemize}
    \item Вероятности~$P_z(\pi)$ задают распределение на множестве всех перестановок~$n_q$ элементов.
    \item Пусть перестановка~$\pi$ ставит объект~$x_i$ выше объекта~$x_j$,
        и~$z_i > z_j$.
        Если поменять эти объекты местами в перестановке~(то есть поставить~$x_j$ выше, чем~$x_i$),
        то новая перестановка будет иметь меньшую вероятность, чем старая.
        Иными словами, чем ближе перестановка к оптимальной, тем выше её вероятность.
    \item Максимальную вероятность имеет перестановка, которая сортирует объекты по убыванию~$z_i$;
        минимальную вероятность имеет обратная к ней перестановка.
\end{itemize}

Таким образом, данные вероятности отдают предпочтение тем перестановкам, которые ближе
к сортировке объектов по предсказаниям алгоритма.
Значит, мы действительно получим способ~<<сглаживания>> ранжирования документов по прогнозам~$z_i$.

Всего перестановок объектов~$n_q!$, и посчитать по ним всем матожидание не представляется возможным.
Чтобы упростить подсчёты, рассмотрим вероятность~$P_z(j)$ попадания объекта~$x_j$ на первое место
после перестановки.
Можно показать, что они вычисляются по формуле
\[
    P_z(j)
    =
    \frac{
        \phi(z_j)
    }{
        \sum_{k = 1}^{n}
            \phi(z_k)
    }
\]
Данные вероятности образуют распределение на всех~$n_q$ документах.
Также для объектов с~$z_i > z_j$ выполнено~$P_z(i) > P_z(j)$~---
то есть введённые вероятности задают на объектах порядок, совпадающий
с ранжированием по оценкам модели.

Теперь мы можем сравнить истинное ранжирование документов и ранжирование по оценкам модели,
посчитав кросс-энтропию между соответствующими им распределениями:
\[
    Q(y, z)
    =
    -
    \sum_{j = 1}^{n_q}
        P_y(j)
        \log P_z(j).
\]
Если взять, например,~$\phi(x) = \exp(x)$,
то кросс-энтропия будет дифференцируема по оценкам модели~---
а значит, можно найти производные и по параметрам модели.
Благодаря сглаживанию мы смогли ввести функционал, который отражает
требования к перестановке объектов,
и при этом позволяет обучение градиентными методами.


%\section{Разметка данных}

%\section{Сравнение моделей ранжирования в онлайн-экспериментах}

\end{document}
