\documentclass[12pt,fleqn]{article}
\usepackage{../lecture-notes/vkCourseML}

\theorembodyfont{\rmfamily}
\newtheorem{esProblem}{Задача}

\title{Машинное обучение, ФКН ВШЭ\\Теоретическое домашнее задание №7}
\author{}
\date{}

\begin{document}
\maketitle

\begin{esProblem}
    Покажите, что параметр $\nu$, используемый в постановке задачи одноклассового SVM,
    является верхней оценкой на долю аномалий на обучающей выборке, т.е. что
    $\frac{1}{\ell} \sum_{i=1}^\ell [\langle w, x_i \rangle < \rho] \leq \nu$, где $w$~--- оптимальный вектор весов.
\end{esProblem}

\begin{esProblem}
    В факторизационных машинах предсказание для объекта~$x \in \RR^{d}$ делается по формуле
    \[
        a(x)
        =
        w_0
        +
        \sum_{j = 1}^{d}
            w_j x_j
        +
        \sum_{j = 1}^{d} \sum_{k = j + 1}^{d}
            x_j x_k
            \langle v_j, v_k \rangle,
    \]
    где $w_j \in \RR, v_j \in \RR^r, j = \overline{0, d}$.
    Вычисление предсказания по этой формуле требует~$O(rd^2)$ операций. Покажите, что это же предсказание может быть сделано за~$O(rd)$ операций.
\end{esProblem}

\begin{esProblem}
    Допустим, мы решили задать правдоподобие выборки через нормальное распределение для размеченных объектов
    и через смесь гауссиан для неразмеченных:
    \begin{align*}
        \sum_{i = 1}^{\ell}
            \log \pi_{y_i} &p(x_i \cond y_i, \theta)\\
        &+
        \sum_{i = \ell + 1}^{\ell + k}
        \log \sum_{y \in \YY}
            \pi_y p(x_i \cond y, \theta)
        \to
        \max_\theta,
    \end{align*}
    где~$p(x \cond y, \theta) = \NN(x \cond \mu_y, \Sigma_y)$.
    Введите скрытые переменные и запишите формулы для E- и M-шагов EM-алгоритма.
\end{esProblem}

\begin{esProblem}
    Рассмотрим нормированный лапласиан~--- разновидность лапласиана, часто возникающую в литературе:
    \[
        L_n
        =
        I - D^{-1/2} W D^{-1/2}.
    \]
    Покажите, что для любого вектора~$f \in \RR^\ell$ выполнено
    \[
        f^T L_n f
        =
        \frac12
        \sum_{i = 1}^{\ell}
            w_{ij}
            \left(
                \frac{f_i}{\sqrt{d_i}}
                -
                \frac{f_j}{\sqrt{d_j}}
            \right)^2.
    \]
\end{esProblem}


\end{document} 
